\capitulo{3}{Conceptos teóricos}
\section{Encriptado de contraseñas}
Para cifrar las contraseñas se estudiaron las posibles funciones de encriptación para este proyecto.
\subsection{password\_hash}
Esta función de PHP crea un hash de contraseña usando un algoritmo de hash fuerte de único sentido.\\
Una función criptográfica hash es un algoritmo matemático que transforma cualquier bloque arbitrario de datos en una nueva serie de caracteres con una longitud fija. Independientemente de la longitud de los datos de entrada, el valor hash de salida tendrá siempre la misma longitud.\cite{definicionHash}
Password\_hash() se emplea para crear un hash con una cadena dada como primer argumento utilizando el algoritmo pasado como segundo argumento. 
Se puede elegir entre estos dos algoritmos para el encriptado:
\begin{description}
    \item [PASSWORD\_DEFAULT] Usa el algoritmo bcrypt. Esta constante está diseñada para cambiar siempre que se añada un algoritmo nuevo y más fuerte a PHP. Por esto la longitud de la encriptación puede variar según cambia el algoritmo.
    \item [PASSWORD\_BCRYPT] Usa el algoritmo CRYPT\_BLOWFISH para crear el hash. Produce un hash usando el identificador "\$2y\$". El resultado siempre será un string de 60 caracteres, o FALSE en caso de error.
\end{description}
Para comprobar que una contraseña introducida en el login coincide con el hash guardado en la base de datos se usa la función password\_verify().\cite{passwordHash}
\imagen{imagenes/Ejpasshash}{Ejemplo de la utilización de password\_hash()}
\subsection{crypt}
Esta función tiene un funcionamiento muy similar que password\_hash().\\
Se le pasan dos parámetros, el string que queremos encriptar y un string de salt para la base del hash.\cite{crypt} A continuación se define lo que es un string salt.\\
En criptografía, la sal o salt en inglés comprende bits aleatorios que se usan como argumento en una función que crea contraseñas. El otro argumento es el string que queremos codificar. El string salt también puede usarse como parte de una clave en un cifrado u otro algoritmo criptográfico. A veces se usa como salt el vector de inicialización, un valor generado previamente.
Las claves con sal complican los ataques de diccionario que cifran cada una de las entradas del mismo: cada bit de sal duplica la cantidad de almacenamiento y computación requeridas.\cite{salt}
\subsection{Hash de Laravel}
Laravel tiene una clase Hash que permite el encriptado de contraseñas usando los algoritmos Bcrypt y Argon2.\\
Bcrypt es una excelente opción para hacer hash de contraseñas porque su factor de trabajo es ajustable, lo que significa que el tiempo que lleva generar un hash puede incrementarse a medida que aumenta la potencia del hardware. Cuando se procesan contraseñas, la lentitud es buena. Cuanto más tiempo tarda un algoritmo en codificar una contraseña, más tardan los usuarios malintencionados en generar todas las cadenas posibles que pueden utilizarse en ataques de fuerza bruta contra aplicaciones.\cite{HashLaravel}
\imagen{imagenes/HashLaravel}{Encriptación de contraseñas en mi aplicación}
Después de informarme de todas las posibilidades que había para el cifrado, elegí la clase hash de Laravel usando el algoritmo bcrypt.\\
Escogí esta porque me pareció la más actual y documentada y, además pertenece a Laravel por lo que pensé que sería mas adecuada para mi trabajo. En cuanto al algoritmo bcrypt me pareció bastante fiable y eficaz al utilizarse en otros sistema de elevada relevancia como OpenBSD y algunas distribuciones de Linux.
 \section{Datos abiertos}
\textit{Los datos abiertos son datos que pueden ser utilizados, reutilizados y redistribuidos libremente por cualquier persona, y que se encuentran sujetos, cuando más, al requerimiento de atribución y de compartirse de la misma manera en que aparecen.}\cite{datosabiertos}\\
Los datos abiertos:
\begin{itemize}
    \item Deben ser abiertos jurídicamente hablando, es decir, que deben estar en un sitio de acceso público y que sus condiciones de uso sean libres y sin restricciones.
    \item Tienen que ser publicados en formatos que puedan ser leídos por dispositivos electrónicos. Además deben permitir que su acceso sea universal y gratuito para todos los usuarios, sin el uso de contraseñas, restricciones o \textit{firewalls}.
\end{itemize}
Existen dos conceptos importantes respecto a los datos abiertos:
\begin{description}
    \item [Catálogo de datos] Lista de conjuntos de datos disponibles en la plataforma de datos abiertos a la que estamos accediendo. Suele componerse de metadatos, información de la licencia de uso y datos. Es el elemento más importante en la plataforma de distribución de datos abiertos. Pueden ofrecerse en varios formatos como JSON, XML, CSV, etc.
    \item[Plataforma] Servicio que da acceso a los usuarios al catálogo de datos. Además suele ofrecer un foro en línea para posibles preguntas, apoyo técnico, etc.
\end{description}
Algunas plataformas de datos abiertos poseen APIs (Application Programming Interfaces). Estas APIs nos permiten comunicarnos con el catálogo de datos de una forma más dinámica y precisa. 
\section{Algoritmos de predicción de datos}
Uno de los requisitos que se propuso al iniciar el proyecto fue el incluir algún tipo de predicción o proyección de datos a futuro, empleando para ello algún algoritmo básico de Machine Learning. A continuación se describen algunos conceptos de la minería de datos.
Los algoritmos de predicción de datos se distinguen en dos grandes grupos: Clasificadores y regresores.
\subsection{Clasificadores}
Estos algoritmos predicen datos categóricos basados en la información con la que han sido entrenados. No entraré en más detalle ya que no se usan en el proyecto.
\subsection{Regresores}
Los estimadores de tipo regresor se usan para predecir datos continuos. Son los siguientes:
\begin{description}
    \item [Adaline] Es un tipo de red neuronal artificial caracterizada por tener múltiples nodos los cuales aceptan muchos inputs para generar un solo output \cite{Adaline}. Las variables que utiliza son:
    \begin{description}
        \item [x] Es el vector que contiene los datos de entrada.
        \item [w] Vector que indica la fuerza de conexión entre los valores de entrada y la neurona.
        \item [n] Número de inputs.
        \item [\begin{math}\theta \end{math}.] La constante.
        \item [y] Datos de salida.
    \end{description}
	\item [Regression Tree] Consiste en un árbol de decisión que se va bifurcando según los posibles resultados de una serie de decisiones relacionadas. Los parámetros son los siguientes:
	\begin{description}
	    \item [maxHeight] Altura máxima del árbol.
	    \item [maxLeafSize] Máximo número de decisiones que una hoja puede tener.
	    \item [maxFeatures] Máximo número de columnas a considerar al elegir una decisión.
	    \item [minPurityIncrease] Aumento mínimo de pureza necesario para que un nodo no se pode durante el crecimiento del árbol.
	\end{description}
	\item [Extra Tree Regressor] Es igual que el anterior pero se diferencian en que este modelo escoge la siguiente decisión de forma aleatoria en vez de buscar el mejor valor en una serie de datos. Son muy rápidos de construir pero sus resultados son muy variables. Los parámetros son los mismos.
	\item [KNN] K Nearest Neighbors (KNN) es un algoritmo de fuerza bruta que localiza el k más cercano de los valores de entrada con los que se ha entrenado para hacer su predicción. Tiene tres parámetros:
	\begin{description}
	    \item[k] El número de vecinos a considerar al hacer la predicción.
	    \item[weighted] Booleano que indica si se debe considerar la distancia de los vecinos más cercanos al hacer la predicción.
	    \item[kernel] El kernel que se va a usar para computar la distancia entre los puntos de referencia.
	\end{description}
	\item[K-d Neighbors Regressor] Es una implementación rápida del algoritmo anterior. Usa un árbol binario con reconocimiento espacial para la búsqueda de vecinos más cercanos. Kd Neighbors Regressor funciona localizando la vecindad de una muestra mediante una búsqueda binaria y luego realiza una búsqueda de fuerza bruta solo en las muestras cercanas o dentro de la vecindad de la muestra desconocida.  La principal ventaja de Kd Neighbors sobre la fuerza bruta KNN es la velocidad de inferencia. Parámetros:
	\begin{description}
	    \item[k] El número de vecinos a considerar al hacer la predicción.
	    \item[weighted] Booleano que indica si se debe considerar la distancia de los vecinos más cercanos al hacer la predicción.
	    \item[tree] Árbol usado para ejecutar las búsquedas de los vecinos cercanos.
	\end{description}
	\item [MLP Regressor] Es una red neuronal multicapa con un output continuo adecuado para problemas de regresión.  El regresor de perceptrón multicapa (MLP) es capaz de manejar problemas complejos de regresión no lineal formando representaciones de orden superior de las características de entrada utilizando capas ocultas intermedias definidas por el usuario.\\
	El MLP también tiene instantáneas de red y monitoreo del progreso.
\end{description}




 