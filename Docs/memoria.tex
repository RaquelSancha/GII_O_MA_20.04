\documentclass[a4paper,12pt,twoside]{memoir}
% Castellano
\usepackage[spanish,es-tabla]{babel}
\selectlanguage{spanish}
\usepackage[utf8]{inputenc}
\usepackage[T1]{fontenc}
\usepackage{lmodern} % Scalable font
\usepackage{microtype}
\usepackage{placeins}
\usepackage[numbers,sort]{natbib}

\RequirePackage{booktabs}
\RequirePackage[table]{xcolor}
\RequirePackage{xtab}
\RequirePackage{multirow}

% Links
\usepackage[colorlinks]{hyperref}
\hypersetup{
	allcolors = {red}
}

% Ecuaciones
\usepackage{amsmath}

% Rutas de fichero / paquete
\newcommand{\ruta}[1]{{\sffamily #1}}

% Párrafos
\nonzeroparskip


% Imagenes
\usepackage{graphicx}
\newcommand{\imagen}[2]{
	\begin{figure}[!h]
		\centering
		\includegraphics[width=0.9\textwidth]{#1}
		\caption{#2}\label{fig:#1}
	\end{figure}
	\FloatBarrier
}

\newcommand{\imagenflotante}[2]{
	\begin{figure}%[!h]
		\centering
		\includegraphics[width=0.9\textwidth]{#1}
		\caption{#2}\label{fig:#1}
	\end{figure}
}



% El comando \figura nos permite insertar figuras comodamente, y utilizando
% siempre el mismo formato. Los parametros son:
% 1 -> Porcentaje del ancho de página que ocupará la figura (de 0 a 1)
% 2 --> Fichero de la imagen
% 3 --> Texto a pie de imagen
% 4 --> Etiqueta (label) para referencias
% 5 --> Opciones que queramos pasarle al \includegraphics
% 6 --> Opciones de posicionamiento a pasarle a \begin{figure}
\newcommand{\figuraConPosicion}[6]{%
  \setlength{\anchoFloat}{#1\textwidth}%
  \addtolength{\anchoFloat}{-4\fboxsep}%
  \setlength{\anchoFigura}{\anchoFloat}%
  \begin{figure}[#6]
    \begin{center}%
      \Ovalbox{%
        \begin{minipage}{\anchoFloat}%
          \begin{center}%
            \includegraphics[width=\anchoFigura,#5]{#2}%
            \caption{#3}%
            \label{#4}%
          \end{center}%
        \end{minipage}
      }%
    \end{center}%
  \end{figure}%
}

%
% Comando para incluir imágenes en formato apaisado (sin marco).
\newcommand{\figuraApaisadaSinMarco}[5]{%
  \begin{figure}%
    \begin{center}%
    \includegraphics[angle=90,height=#1\textheight,#5]{#2}%
    \caption{#3}%
    \label{#4}%
    \end{center}%
  \end{figure}%
}
% Para las tablas
\newcommand{\otoprule}{\midrule [\heavyrulewidth]}
%
% Nuevo comando para tablas pequeñas (menos de una página).
\newcommand{\tablaSmall}[5]{%
 \begin{table}[h]
  \begin{center}
   \rowcolors {2}{gray!35}{}
   \begin{tabular}{#2}
    \toprule
    #4
    \otoprule
    #5
    \bottomrule
   \end{tabular}
   \caption{#1}
   \label{tabla:#3}
  \end{center}
 \end{table}
}

%
% Nuevo comando para tablas pequeñas (menos de una página).
\newcommand{\tablaSmallSinColores}[5]{%
 \begin{table}[H]
  \begin{center}
   \begin{tabular}{#2}
    \toprule
    #4
    \otoprule
    #5
    \bottomrule
   \end{tabular}
   \caption{#1}
   \label{tabla:#3}
  \end{center}
 \end{table}
}

\newcommand{\tablaApaisadaSmall}[5]{%
\begin{landscape}
  \begin{table}
   \begin{center}
    \rowcolors {2}{gray!35}{}
    \begin{tabular}{#2}
     \toprule
     #4
     \otoprule
     #5
     \bottomrule
    \end{tabular}
    \caption{#1}
    \label{tabla:#3}
   \end{center}
  \end{table}
\end{landscape}
}

%
% Nuevo comando para tablas grandes con cabecera y filas alternas coloreadas en gris.
\newcommand{\tabla}[6]{%
  \begin{center}
    \tablefirsthead{
      \toprule
      #5
      \otoprule
    }
    \tablehead{
      \multicolumn{#3}{l}{\small\sl continúa desde la página anterior}\\
      \toprule
      #5
      \otoprule
    }
    \tabletail{
      \hline
      \multicolumn{#3}{r}{\small\sl continúa en la página siguiente}\\
    }
    \tablelasttail{
      \hline
    }
    \bottomcaption{#1}
    \rowcolors {2}{gray!35}{}
    \begin{xtabular}{#2}
      #6
      \bottomrule
    \end{xtabular}
    \label{tabla:#4}
  \end{center}
}

%
% Nuevo comando para tablas grandes con cabecera.
\newcommand{\tablaSinColores}[6]{%
  \begin{center}
    \tablefirsthead{
      \toprule
      #5
      \otoprule
    }
    \tablehead{
      \multicolumn{#3}{l}{\small\sl continúa desde la página anterior}\\
      \toprule
      #5
      \otoprule
    }
    \tabletail{
      \hline
      \multicolumn{#3}{r}{\small\sl continúa en la página siguiente}\\
    }
    \tablelasttail{
      \hline
    }
    \bottomcaption{#1}
    \begin{xtabular}{#2}
      #6
      \bottomrule
    \end{xtabular}
    \label{tabla:#4}
  \end{center}
}

%
% Nuevo comando para tablas grandes sin cabecera.
\newcommand{\tablaSinCabecera}[5]{%
  \begin{center}
    \tablefirsthead{
      \toprule
    }
    \tablehead{
      \multicolumn{#3}{l}{\small\sl continúa desde la página anterior}\\
      \hline
    }
    \tabletail{
      \hline
      \multicolumn{#3}{r}{\small\sl continúa en la página siguiente}\\
    }
    \tablelasttail{
      \hline
    }
    \bottomcaption{#1}
  \begin{xtabular}{#2}
    #5
   \bottomrule
  \end{xtabular}
  \label{tabla:#4}
  \end{center}
}



\definecolor{cgoLight}{HTML}{EEEEEE}
\definecolor{cgoExtralight}{HTML}{FFFFFF}

%
% Nuevo comando para tablas grandes sin cabecera.
\newcommand{\tablaSinCabeceraConBandas}[5]{%
  \begin{center}
    \tablefirsthead{
      \toprule
    }
    \tablehead{
      \multicolumn{#3}{l}{\small\sl continúa desde la página anterior}\\
      \hline
    }
    \tabletail{
      \hline
      \multicolumn{#3}{r}{\small\sl continúa en la página siguiente}\\
    }
    \tablelasttail{
      \hline
    }
    \bottomcaption{#1}
    \rowcolors[]{1}{cgoExtralight}{cgoLight}

  \begin{xtabular}{#2}
    #5
   \bottomrule
  \end{xtabular}
  \label{tabla:#4}
  \end{center}
}

\graphicspath{ {./imagenes/} }

% Capítulos
\chapterstyle{bianchi}
\newcommand{\capitulo}[2]{
	\setcounter{chapter}{#1}
	\setcounter{section}{0}
	\chapter*{#2}
	\addcontentsline{toc}{chapter}{#2}
	\markboth{#2}{#2}
}

% Apéndices
\renewcommand{\appendixname}{Apéndice}
\renewcommand*\cftappendixname{\appendixname}

\newcommand{\apendice}[1]{
	%\renewcommand{\thechapter}{A}
	\chapter{#1}
}

\renewcommand*\cftappendixname{\appendixname\ }

% Formato de portada
\makeatletter
\usepackage{xcolor}
\newcommand{\tutor}[1]{\def\@tutor{#1}}
\newcommand{\course}[1]{\def\@course{#1}}
\definecolor{cpardoBox}{HTML}{E6E6FF}
\def\maketitle{
  \null
  \thispagestyle{empty}
  % Cabecera ----------------
\noindent\includegraphics[width=\textwidth]{./imagenes/escudoubu}\vspace{1cm}%
  \vfill
  % Título proyecto y escudo informática ----------------
  \colorbox{cpardoBox}{%
    \begin{minipage}{\textwidth}
      \vspace{.5cm}\Large
      \begin{center}
      \textbf{TFG del Grado en Ingeniería Informática}\vspace{.6cm}\\
      \textbf{\LARGE\@title{}}
      \end{center}
      \vspace{.2cm}
    \end{minipage}

  }%

  \vfill
  % Datos de alumno, curso y tutores ------------------
  \begin{center}%
  {%
    \noindent\LARGE
    Presentado por \@author{}\\ 
    en Universidad de Burgos --- \@date{}\\
    Tutores: \@tutor{}\\
  }%
  \end{center}%
  \null
  \cleardoublepage
  }
\makeatother

\newcommand{\nombre}{Raquel Sancha Sánchez} %%% cambio de comando

% Datos de portada
\title{Mejora de la herramienta web dinámica para la
captación, tratamiento y presentación de
datos relacionados con la coyuntura
económica de Burgos}
\author{\nombre}
\tutor{Bruno Baruque Zanon\\
Santiago Porras Alfonso\\
Carlos López Nozal}
\date{\today}

\begin{document}

\maketitle


\newpage\null\thispagestyle{empty}\newpage


%%%%%%%%%%%%%%%%%%%%%%%%%%%%%%%%%%%%%%%%%%%%%%%%%%%%%%%%%%%%%%%%%%%%%%%%%%%%%%%%%%%%%%%%
\thispagestyle{empty}


\noindent\includegraphics[width=\textwidth]{./imagenes/escudoubu}\vspace{1cm}

\noindent D. nombre tutor, profesor del departamento de nombre departamento, área de nombre área.

\noindent Expone:

\noindent Que el alumno D. \nombre, con DNI 71567050C, ha realizado el Trabajo final de Grado en Ingeniería Informática titulado Mejora de la herramienta web dinámica para la captación, tratamiento y presentación de datos relacionados con la coyuntura económica de Burgos. 

\noindent Y que dicho trabajo ha sido realizado por el alumno bajo la dirección del que suscribe, en virtud de lo cual se autoriza su presentación y defensa.

\begin{center} %\large
En Burgos, {\large \today}
\end{center}

\vfill\vfill\vfill

% Author and supervisor
\begin{minipage}{0.45\textwidth}
\begin{flushleft} %\large
Vº. Bº. del Tutor:\\[2cm]
D. nombre tutor
\end{flushleft}
\end{minipage}
\hfill
\begin{minipage}{0.45\textwidth}
\begin{flushleft} %\large
Vº. Bº. del co-tutor:\\[2cm]
D. nombre co-tutor
\end{flushleft}
\end{minipage}
\hfill

\vfill

% para casos con solo un tutor comentar lo anterior
% y descomentar lo siguiente
%Vº. Bº. del Tutor:\\[2cm]
%D. nombre tutor


\newpage\null\thispagestyle{empty}\newpage




\frontmatter

% Abstract en castellano
\renewcommand*\abstractname{Resumen}
\begin{abstract}
Este trabajo es una ampliación de la primera versión del TFG llamado \textit{Herramienta web dinámica para la captación, tratamiento y presentación de datos relacionados con la coyuntura económica de Burgos} \href{https://github.com/NelsonParamo/GI16.M_ProyectoCoyuntura/}{Link al proyecto en Github}.\\
Este proyecto fue creado para facilitar el trabajo del equipo que todos los años realiza unos boletines de coyuntura económica para informar de la situación de la economía burgalesa.\\
Esta aplicación permite la entrada de datos y su clasificación.\\
El objetivo del presente proyecto es realizar mejoras en el diseño de la aplicación y la corrección de errores o malas prácticas del código así como añadir nuevas funcionalidades como la introducción de datos directamente de la página del INE o la capacidad de exportar las tablas a Excel.
\end{abstract}

\renewcommand*\abstractname{Descriptores}
\begin{abstract}
Economía, coyuntura económica, boletín, MySQL, Laravel, PHP, variables, aplicación web, Instituto Nacional de Estadística, predicción

 \ldots
\end{abstract}

\clearpage

% Abstract en inglés
\renewcommand*\abstractname{Abstract}
\begin{abstract}
This work is an extension of the first version of the TFG called \textit {Dynamic web tool for the collection, treatment and presentation of data related to the economic situation of Burgos} \href {https://github.com/NelsonParamo/GI16. M_ProyectoCoyuntura /} {Link to the project on Github}. \\
This project was created to facilitate the work of the team that every year produces economic newsletters to report on the situation of the Burgos economy. \\
This application allows data entry and classification. \\
The objective of this project is to make improvements in the design of the application and the correction of errors or bad practices in the code as well as to add new functionalities such as entering data directly from the INE page or the ability to export the tables to Excel.
\end{abstract}

\renewcommand*\abstractname{Keywords}
\begin{abstract}
Economy, economic situation, newsletter, MySQL, Laravel, PHP, variables, web application, National Institute of Statistics, prediction 
\end{abstract}

\clearpage

% Indices
\tableofcontents

\clearpage

\listoffigures

\clearpage

\listoftables
\clearpage

\mainmatter
\capitulo{1}{Introducción}
Todos los años, el departamento de economía de la Universidad de Burgos realiza un boletín exhaustivo sobre la coyuntura económica en el ámbito burgalés.\\
Para ello recopila datos de distintas fuentes: Instituto Nacional de estadística, banco nacional de España, Eurostat...\\
Este trabajo surge de la necesidad de una herramienta que permita organizar y tratar dichos datos, así como almacenarlos y facilitar el trabajo al departamento que los recoge.\\
El objetivo de la realización de los boletines de coyuntura económica es conocer el desarrollo de la economía del ámbito estudiado, mediante la producción de información económica y su divulgación a una amplio público de empresas, profesionales y particulares promoviendo así el análisis de la coyuntura económica.\\
Estos boletines, como ya he indicado antes, los crea el Equipo multidisciplinar de Coyuntura radicado en la Facultad de Ciencias Económicas y Empresariales de la Universidad de Burgos. En virtud del Convenio Marco de Colaboración firmado por la Universidad de Burgos y la actual Caja Viva Caja Rural.\\
Este equipo multidisciplinar integrado por 16 profesores, de los Departamentos de Economía Aplicada, Economía y Administración de Empresas y Derecho analizan la evolución económica coyuntural de la provincia de Burgos.\\
La aplicación va dirigida al equipo de coyuntura económica, por lo que la mayoría de los usuarios que van a manejar esta aplicación van a ser administradores, debido a esto, se ha creado el rol del super administrador, para tener un control de los administradores ya que estos no son usuarios experimentados.\\
Por lo tanto, la aplicación proporciona diferentes funcionalidades que dependerán del rol (super administrador, administrador y invitado) del usuario que haya accedido.\\
La principal funcionalidad de la aplicación consiste en la introducción de datos estadísticos de la coyuntura económica dependientes de una categoría específica, un ámbito geográfico y un
año.\\
A partir de estos datos se generarán tablas que podrán ser filtradas y mostradas al usuario de la manera que a éste le resulte mas cómoda y utilizarlas para realizar el boletín.\\
La aplicación también genera gráficos que muestran los datos de una forma sencilla e intuitiva.\\
Los invitados que accedan a la aplicación deberán tener un rol solo de lectura, por lo que sólo podrán filtrar y visualizar las tablas y gráficos de las variables económicas que están almacenadas en la base de datos de la aplicación.\\
Los encargados de introducir datos estadísticos, modificarlos o borrar los que estén obsoletos o erróneos, serán los administradores de la aplicación.\\
La aplicación cuenta con una ayuda sobre cómo manejar las funcionalidades de la aplicación, esta ayuda varía dependiendo del rol del usuario conectado.\\
Mi trabajo consiste en mejorar la aplicación, corregir algunos errores y añadir nuevas funcionalidades como la de introducir datos de forma automatizada desde el Instituto Nacional de Estadística (INE).\\



\capitulo{2}{Objetivos del proyecto}
\section{Objetivos generales}
\begin{itemize}
	\item Identificar los errores existentes en la aplicación. 
	\item Corregir los errores. 
	\item Ampliar la funcionalidad de la aplicación permitiendo introducir datos desde el INE.
	\item Realizar predicciones de los datos y mostrarlas en gráficos.
	\item Permitir exportar los datos de las tablas a Excel.
	\item Mejorar la ayuda a los usuarios.
	\item Implementar la edición de usuarios.
	\item Mejorar la seguridad de la aplicación.
	\item Publicar la aplicación en su entorno de explotación final que es el servidor de la UBU.
	\item Crear una aplicación web funcional y operativa que funcione sin errores.
\end{itemize}
\section{Objetivos técnicos}
\begin{itemize}
	\item Aprovechar las ventajas de la arquitectura de Laravel que consiste en modelos, vistas y controladores.
	\item Hacer uso de visual studio code para el desarrollo del código.
	\item Conseguir transferir los datos del INE en formato JSON a las tablas de la base de datos.
	\item Exportar correctamente los datos de las tablas a Excel.
	\item Encriptar las contraseñas usando un algoritmo seguro y eficiente.
	\item Mejorar la administración de usuarios.
\end{itemize}
\section{Objetivos personales}
\begin{itemize}
	\item Aprender a programar en PHP.
	\item Aprender la metodología de Laravel y de las aplicaciones web.
	\item Entender el código escrito por otra persona y modificarlo según las necesidades para realizar procedimientos comunes en el desarrollo de software profesional, como el mantenimiento del software, refactorización de código, etc.
	
\end{itemize}
\capitulo{3}{Conceptos teóricos}
\section{Encriptado de contraseñas}
Para cifrar las contraseñas se estudiaron las posibles funciones de encriptación para este proyecto.
\subsection{password\_hash}
Esta función de PHP crea un hash de contraseña usando un algoritmo de hash fuerte de único sentido.\\
Una función criptográfica hash es un algoritmo matemático que transforma cualquier bloque arbitrario de datos en una nueva serie de caracteres con una longitud fija. Independientemente de la longitud de los datos de entrada, el valor hash de salida tendrá siempre la misma longitud.\cite{definicionHash}
Password\_hash() se emplea para crear un hash con una cadena dada como primer argumento utilizando el algoritmo pasado como segundo argumento. 
Se puede elegir entre estos dos algoritmos para el encriptado:
\begin{description}
    \item [PASSWORD\_DEFAULT] Usa el algoritmo bcrypt. Esta constante está diseñada para cambiar siempre que se añada un algoritmo nuevo y más fuerte a PHP. Por esto la longitud de la encriptación puede variar según cambia el algoritmo.
    \item [PASSWORD\_BCRYPT] Usa el algoritmo CRYPT\_BLOWFISH para crear el hash. Produce un hash usando el identificador "\$2y\$". El resultado siempre será un string de 60 caracteres, o FALSE en caso de error.
\end{description}
Para comprobar que una contraseña introducida en el login coincide con el hash guardado en la base de datos se usa la función password\_verify().\cite{passwordHash}
\imagen{imagenes/Ejpasshash}{Ejemplo de la utilización de password\_hash()}
\subsection{crypt}
Esta función tiene un funcionamiento muy similar que password\_hash().\\
Se le pasan dos parámetros, el string que queremos encriptar y un string de salt para la base del hash.\cite{crypt} A continuación se define lo que es un string salt.\\
En criptografía, la sal o salt en inglés comprende bits aleatorios que se usan como argumento en una función que crea contraseñas. El otro argumento es el string que queremos codificar. El string salt también puede usarse como parte de una clave en un cifrado u otro algoritmo criptográfico. A veces se usa como salt el vector de inicialización, un valor generado previamente.
Las claves con sal complican los ataques de diccionario que cifran cada una de las entradas del mismo: cada bit de sal duplica la cantidad de almacenamiento y computación requeridas.\cite{salt}
\subsection{Hash de Laravel}
Laravel tiene una clase Hash que permite el encriptado de contraseñas usando los algoritmos Bcrypt y Argon2.\\
Bcrypt es una excelente opción para hacer hash de contraseñas porque su factor de trabajo es ajustable, lo que significa que el tiempo que lleva generar un hash puede incrementarse a medida que aumenta la potencia del hardware. Cuando se procesan contraseñas, la lentitud es buena. Cuanto más tiempo tarda un algoritmo en codificar una contraseña, más tardan los usuarios malintencionados en generar todas las cadenas posibles que pueden utilizarse en ataques de fuerza bruta contra aplicaciones.\cite{HashLaravel}
\imagen{imagenes/HashLaravel}{Encriptación de contraseñas en mi aplicación}
Después de informarme de todas las posibilidades que había para el cifrado, elegí la clase hash de Laravel usando el algoritmo bcrypt.\\
Escogí esta porque me pareció la más actual y documentada y, además pertenece a Laravel por lo que pensé que sería mas adecuada para mi trabajo. En cuanto al algoritmo bcrypt me pareció bastante fiable y eficaz al utilizarse en otros sistema de elevada relevancia como OpenBSD y algunas distribuciones de Linux.
 \section{Datos abiertos}
\textit{Los datos abiertos son datos que pueden ser utilizados, reutilizados y redistribuidos libremente por cualquier persona, y que se encuentran sujetos, cuando más, al requerimiento de atribución y de compartirse de la misma manera en que aparecen.}\cite{datosabiertos}\\
Los datos abiertos:
\begin{itemize}
    \item Deben ser abiertos jurídicamente hablando, es decir, que deben estar en un sitio de acceso público y que sus condiciones de uso sean libres y sin restricciones.
    \item Tienen que ser publicados en formatos que puedan ser leídos por dispositivos electrónicos. Además deben permitir que su acceso sea universal y gratuito para todos los usuarios, sin el uso de contraseñas, restricciones o \textit{firewalls}.
\end{itemize}
Existen dos conceptos importantes respecto a los datos abiertos:
\begin{description}
    \item [Catálogo de datos] Lista de conjuntos de datos disponibles en la plataforma de datos abiertos a la que estamos accediendo. Suele componerse de metadatos, información de la licencia de uso y datos. Es el elemento más importante en la plataforma de distribución de datos abiertos. Pueden ofrecerse en varios formatos como JSON, XML, CSV, etc.
    \item[Plataforma] Servicio que da acceso a los usuarios al catálogo de datos. Además suele ofrecer un foro en línea para posibles preguntas, apoyo técnico, etc.
\end{description}
Algunas plataformas de datos abiertos poseen APIs (Application Programming Interfaces). Estas APIs nos permiten comunicarnos con el catálogo de datos de una forma más dinámica y precisa. 
\section{Algoritmos de predicción de datos}
Uno de los requisitos que se propuso al iniciar el proyecto fue el incluir algún tipo de predicción o proyección de datos a futuro, empleando para ello algún algoritmo básico de Machine Learning. A continuación se describen algunos conceptos de la minería de datos.
Los algoritmos de predicción de datos se distinguen en dos grandes grupos: Clasificadores y regresores.
\subsection{Clasificadores}
Estos algoritmos predicen datos categóricos basados en la información con la que han sido entrenados. No entraré en más detalle ya que no se usan en el proyecto.
\subsection{Regresores}
Los estimadores de tipo regresor se usan para predecir datos continuos. Son los siguientes:
\begin{description}
    \item [Adaline] Es un tipo de red neuronal artificial caracterizada por tener múltiples nodos los cuales aceptan muchos inputs para generar un solo output \cite{Adaline}. Las variables que utiliza son:
    \begin{description}
        \item [x] Es el vector que contiene los datos de entrada.
        \item [w] Vector que indica la fuerza de conexión entre los valores de entrada y la neurona.
        \item [n] Número de inputs.
        \item [\begin{math}\theta \end{math}.] La constante.
        \item [y] Datos de salida.
    \end{description}
	\item [Regression Tree] Consiste en un árbol de decisión que se va bifurcando según los posibles resultados de una serie de decisiones relacionadas. Los parámetros son los siguientes:
	\begin{description}
	    \item [maxHeight] Altura máxima del árbol.
	    \item [maxLeafSize] Máximo número de decisiones que una hoja puede tener.
	    \item [maxFeatures] Máximo número de columnas a considerar al elegir una decisión.
	    \item [minPurityIncrease] Aumento mínimo de pureza necesario para que un nodo no se pode durante el crecimiento del árbol.
	\end{description}
	\item [Extra Tree Regressor] Es igual que el anterior pero se diferencian en que este modelo escoge la siguiente decisión de forma aleatoria en vez de buscar el mejor valor en una serie de datos. Son muy rápidos de construir pero sus resultados son muy variables. Los parámetros son los mismos.
	\item [KNN] K Nearest Neighbors (KNN) es un algoritmo de fuerza bruta que localiza el k más cercano de los valores de entrada con los que se ha entrenado para hacer su predicción. Tiene tres parámetros:
	\begin{description}
	    \item[k] El número de vecinos a considerar al hacer la predicción.
	    \item[weighted] Booleano que indica si se debe considerar la distancia de los vecinos más cercanos al hacer la predicción.
	    \item[kernel] El kernel que se va a usar para computar la distancia entre los puntos de referencia.
	\end{description}
	\item[K-d Neighbors Regressor] Es una implementación rápida del algoritmo anterior. Usa un árbol binario con reconocimiento espacial para la búsqueda de vecinos más cercanos. Kd Neighbors Regressor funciona localizando la vecindad de una muestra mediante una búsqueda binaria y luego realiza una búsqueda de fuerza bruta solo en las muestras cercanas o dentro de la vecindad de la muestra desconocida.  La principal ventaja de Kd Neighbors sobre la fuerza bruta KNN es la velocidad de inferencia. Parámetros:
	\begin{description}
	    \item[k] El número de vecinos a considerar al hacer la predicción.
	    \item[weighted] Booleano que indica si se debe considerar la distancia de los vecinos más cercanos al hacer la predicción.
	    \item[tree] Árbol usado para ejecutar las búsquedas de los vecinos cercanos.
	\end{description}
	\item [MLP Regressor] Es una red neuronal multicapa con un output continuo adecuado para problemas de regresión.  El regresor de perceptrón multicapa (MLP) es capaz de manejar problemas complejos de regresión no lineal formando representaciones de orden superior de las características de entrada utilizando capas ocultas intermedias definidas por el usuario.\\
	El MLP también tiene instantáneas de red y monitoreo del progreso.
\end{description}




 
\capitulo{4}{Técnicas y herramientas}
\section{Técnicas}
\subsection{Scrum}
En Scrum un proyecto se ejecuta en ciclos temporales cortos y de duración fija (iteraciones o sprints).\cite{scrum:definicion}\\
A cada sprint se le asigna unas tareas, generalmente con relación entre sí y en torno a un mismo tema.\\ 
Al finalizar la iteración se revisan las tareas en una reunión con el resto de miembros del equipo.. 
\section{Herramientas}
Voy a clasificar las herramientas según si las he usado en el desarrollo del proyecto, para probarlo, para evaluar la calidad del código o para redactar la documentación.  
\subsection{Desarrollo}
\subsubsection{Laravel}
El framework usado para el desarrollo de la aplicación ha sido Laravel.\\
Laravel incluye Eloquent, un mapeador relacional de objetos (ORM) que simplifica la interacción con la base de datos. Cuando se usa Eloquent, cada tabla de la base de datos tiene un modelo correspondiente que se usa para interactuar con esa tabla. Además de recuperar registros de la tabla de la base de datos, los modelos Eloquent también le permiten insertar, actualizar y eliminar registros de la tabla.\cite{Laravel:modelos}\\
Laravel también se caracteriza por el uso de controladores para agrupar la funcionalidad de un determinado recurso.\cite{Laravel:controladores}\\
Posee una estructura de directorios predefinidos que ayuda a organizar nuestro proyecto. Los más importantes son \textit{resources} que contiene las vistas de la aplicación, \textit{routes} que tiene el archivo que guarda la definición de las rutas y \textit{app} que guarda los modelos y los controladores entre otros archivos de configuración del proyecto.\\

\subsubsection{Composer}
Composer es un sistema de gestión de paquetes para programar en PHP el cual provee los formatos estándar necesarios para manejar dependencias y librerías de PHP. \cite{Composer:definicion}\\
Es un gestor de dependencias para proyectos escritos en el leguaje de programación PHP. Eso quiere decir que nos permite gestionar (declarar, descargar y mantener actualizados) los paquetes de software en los que se basa nuestro proyecto PHP.\\
Cuando se empieza un proyecto en PHP, ya de cierta complejidad, no  vale solo con la librería de funciones nativa de PHP. Generalmente se usa alguna que otra librería de terceros desarrolladores, que permite evitar empezar todo desde cero. Ya sea un framework o algo más específico como un sistema para debug o envío de email, registro de usuarios, exportación de datos, etc., cualquier cosa que se pueda necesitar ya puede estar creada por otros desarrolladores.\\
Para usar composer debemos tener un archivo JSON en el que deberemos escribir los paquetes que queramos instalar, el nombre del proyecto, la descripción, algunos comandos que queramos que se ejecuten en el momento de la instalación o de la actualización (como por ejemplo generar la clave del proyecto u optimizarlo), etc. Puede haber muchas opciones de configuración posibles en este archivo. 
El archivo JSON debe llamarse composer.json.\\
Al usar el comando \textit{composer install}, se crea automáticamente un archivo llamado \textit{composer.lock} donde aparece una información más detallada de las dependencias instaladas además de ir a los repositorios de paquetes de software y descargar aquellas librerías mencionadas, copiándolas en la carpeta del proyecto.\\
Estas dependencias se instalan en la carpeta \textit{vendor} de la aplicación.\cite{composer}
\subsubsection{Visual Studio Code}
Es un entorno de desarrollo que permite crear sitios y aplicaciones web. Es compatible con múltiples lenguajes de programación, tales como C++, C\#, Visual Basic .NET, F\#, Java, Python, Ruby y PHP (Éste último es el que se ha usado para el proyecto).\cite{VisualStudio} 
\subsubsection{Git}
Se ha usado Git como herramienta para el control de las versiones del proyecto. Además se puede usar junto con Github desde Visual Studio Code descargándonos su extensión.

\subsection{Entorno de pruebas}

\subsubsection{Laravel dusk}
Laravel Dusk proporciona una API de prueba y automatización para crear y ejecutar tests y así probar nuestra aplicación en laravel.
En la documentación de laravel dusk pone que no hace falta descargarse JDK o Selenium sino que usa una instalación independiente de Chromedriver pero para que a mí me funcionara laravel dusk he tenido que ejecutar los test en un contenedor de Docker con Selenium.\cite{LaravelDusk}

\subsubsection{Xampp}
XAMPP es un paquete de software libre, que consiste principalmente en el sistema de gestión de bases de datos MySQL, el servidor web Apache y los intérpretes para lenguajes de script PHP y Perl. El nombre es en realidad un acrónimo: X (para cualquiera de los diferentes sistemas operativos), Apache, MariaDB/MySQL, PHP, Perl.\cite{Xampp}
He usado esta aplicación para las pruebas locales de mi proyecto.
\subsubsection{Docker}
Docker sirve para automatizar el despliegue de aplicaciones dentro de contenedores software de forma que se pueda probar la aplicación en distintos equipos sin la necesidad de instalar un servidor local como Xampp.\\ \cite{Docker}
Docker tiene como ventaja que aisla solo los recursos del sistema operativo que necesita y no una cantidad de recursos fija como hace una máquina virtual.\\
Para el despliegue de mi aplicación he utilizado un contenedor especial para aplicaciones de Laravel llamado Laradock.\\
Laradock contiene multitud de imágenes conectadas entre ellas que se usan para el despliegue de la app. Sin embargo, yo solo he usado Apache, PHP, PhpMyAdmin, MySQL y Selenium.\\
\subsubsection{Virtual Box}
Al principio usé la máquina virtual Virtual Box para probar mi aplicación en un entorno de Linux (yo trabajo con Windows) ya que el servidor de la ubu al que va destinado el proyecto está en Linux pero no me funcionaba del todo bien así que hice una partición de disco de ubuntu para probarlo en ese entorno. 
\subsubsection{Heroku}
Heroku es una plataforma en la nube que permite subir aplicaciones para probarlas. La principal ventaja que tiene Heroku es que su uso a un nivel básico es gratuito. Como desventaja diría que la aplicación va muy lenta en este servidor web.
\subsection{Calidad del código}
\subsubsection{Codacy}
Para evaluar la calidad de mi código he usado la herramienta Codacy.\\
Permite realizar un análisis del proyecto y reportar los posibles problemas que haya. En el apartado \textit{Issues} se pueden ver clasificados por varios criterios: 
\begin{itemize}
    \item Lenguaje de programación.
    \item Categorías: Estilo del código, seguridad y código sin usar.
    \item Nivel de peligrosidad: Distingue entre errores y warnings.
    \item Patrones no recomendados: Accesos estáticos, nombres de variables cortos, etc.
\end{itemize}
En el apartado \textit{Files} aparecen los archivos del proyecto calificados con una nota en función del número de \textit{Issues} que tenga, el grado de duplicación y la complejidad.
También permite ver la calidad de los cambios desde el apartado \textit{Commits}.\\
Esta aplicación es muy fácil de usar y se puede conectar a un repositorio de Github de manera que al hacer algún cambio en éste el análisis se actualice.\\
 \imagen{imagenes/codacy}{Opciones que ofrece codacy para el análisis de nuestro código}
\subsection{Documentación}
\subsubsection{Latex}
\LaTeX{} es una herramienta para la composición de texto con una serie de comandos que permiten formar tus documentos a tu gusto.\\
Está formado por un gran conjunto de macros de TeX, con la intención de facilitar el uso del lenguaje de composición tipográfica.\cite{wiki:latex}\\
Nunca había usado esta herramienta ni había oído hablar de ella pero estaba en las sugerencias de la ubu para escribir la memoria del trabajo final de grado y me pareció interesante usarla. 
\section{Librerías}
\subsection{Laravel Excel}
Para exportar los datos de las tablas a Excel he usado la biblioteca Laravel Excel.
Esta biblioteca está basada en PhpSpreadsheet que es un recurso escrito en PHP puro el cual proporciona un conjunto de clases que permiten leer y escribir en diferentes formatos de archivo de hoja de cálculo, como Excel y LibreOffice Calc.\cite{LaravelExcel}\\
La he elegido porque me parece fácil de usar y con muchas opciones: estilo de las columnas, exportación de array enteros... además de que la documentación es muy buena y tiene muchos ejemplos.
\subsection{Chart js}
Esta librería en JavaScript es gratuita y de código abierto y se usa para la visualización de datos en forma de gráfico.
Tipos de gráficos que permite crear:
\begin{itemize}
    \item Gráfico de barras.
    \item Gráfico de líneas.
    \item Área.
    \item Burbuja.
    \item Radar.
    \item Polar.
    \item Dispersión.
\end{itemize}
\subsection{Laravel DebugBar}
Este proyecto, realizado por un particular llamado \href{https://github.com/barryvdh}{barryvdh} consiste en una DebugBar elaborada exclusivamente para proyectos en Laravel.\\
Ayuda a identificar errores y a recopilar información de ejecución de la aplicación. Algunos de los datos que recopila son:
\begin{itemize}
    \item Resultado de las consultas realizadas a la base de datos.
    \item Información de la ruta en la que estamos.
    \item Información de las vistas cargadas.
    \item Eventos.
    \item Información de la versión de Laravel y del entorno de desarrollo.
    \item Datos de los usuarios.
    \item Valores de los archivos de configuración.
\end{itemize}
\imagen{imagenes/debugbar}{DebugBar para Laravel}
\subsection{Rubix ML}
Esta biblioteca ha sido usada para la predicción de los datos. Incluye múltiples algoritmos de aprendizaje tanto para datos categóricos como continuos. En el caso de mi aplicación he usado los algoritmos para datos continuos.\\
Las fases por las que tienen que pasar los datos para realizar la predicción son:
\begin{itemize}
    \item Extracción.
    \item Transformación.
    \item Carga.
    \item Entrenamiento.
    \item Predicción.
\end{itemize}
Para realizar la estimación, hay que precargar los datos que tenemos, adaptarlos y pasárselos al estimador para entrenarlo. Los posibles algoritmos que se usan para la estimación los he explicado en el apartado de conceptos teóricos.

\capitulo{5}{Aspectos relevantes del desarrollo del proyecto}
En este apartado se van a recoger los aspectos más importantes del desarrollo del proyecto. Desde cuestiones del desarrollo del proyecto, hasta los numerosos problemas a los que hubo que enfrentarse y cómo se solucionaron.
\section{Inicio del proyecto: Aprendizaje}
En esta fase, probablemente la más costosa a nivel personal, detallaré los conceptos que me hicieron falta aprender para poder empezar mi trabajo.\\
\subsection{Aprendizaje de Laravel} 
Laravel tiene un sistema de carpetas y archivos de configuración muy extenso que al principio cuesta entender pero por suerte existen multitud de tutoriales en forma de vídeos o de artículos que explican como iniciarse en Laravel.\\
Si no se tiene ningún conocimiento previo sobre este framework no recomendaría mirar la documentación de la página oficial ya que, aunque es muy completa, no considero que ofrezca una visión general sobre el funcionamiento del mismo sino información muy específica que no sirve al usuario si no conoce primero la dinámica general de la herramienta.\\
Tutoriales que seguí:
\begin{itemize}
    \item \textit{Laravel PHP Framework Tutorial}(freeCodeCamp)\cite{TutorialLaravel}
    \item \textit{Curso Laravel}(pildorasinformaticas)\cite{TutorialLaravel2}
    \item \textit{Ejemplo de un proyecto en laravel}(Julio Yáñez Novo)\cite{PrimerProyecto}
\end{itemize}
\subsection{Aprendizaje de PHP} 
Este lenguaje no se enseña en la carrera y tampoco había trabajado con él antes así que tuve que aprenderlo desde cero. No me resultó tan difícil como aprender a entender proyectos con Laravel así que esta parte me supuso menos trabajo.\\
Las herramientas que usé para familiarizarme con este lenguaje fueron:
\begin{itemize}
    \item \textit{Tutorial de PHP para principiantes}(ionos)\cite{TutorialPHP}
    \item \textit{Curso php}(tutorialesya)\cite{CursoPHP} 
\end{itemize}
\section{Gestión de los errores}
En la aplicación original existían varios errores que hubo que solucionar. 
\begin{description}
    \item [Exportación de datos a excel]. Existía un botón para exportar los datos de la tabla elegida a Excel pero no funcionaba. Para conseguir que el botón cumpliera la funcionalidad prometida he usado la biblioteca Laravel Excel (explicada anteriormente en el apartado "Técnicas y herramientas")
    \item[Administración de usuarios] Esta sección daba un error cuando se intentaba acceder a ella.\\
    Para entender esta sección hace falta conocer el sistema de usuarios que usa la aplicación:
    \begin{description}
        \item [Súper administrador] Las personas con este rol pueden controlar los roles de los demás usuarios y modificarlos.
        Además se les permite aceptar o declinar las peticiones de registro en la aplicación a nuevos usuarios. También tienen acceso al resto las funcionalidades de la aplicación.
        \item [Administrador] Este tipo de usuarios tienen acceso a todas las partes de la aplicación menos a la de la administración de usuarios, es decir, no pueden aceptar nuevos usuarios ni cambiar roles.
        \item [Invitado] Este tipo de usuarios no está registrado en la aplicación. Sólo tiene acceso a las tablas para su visualización. Para poder acceder a las demás funcionalidades deberá realizar una petición de registro.
    \end{description}
    \imagen{imagenes/Administracionusuarios}{Sección Administración de usuarios}
    Desde esta pantalla, la cual solo se puede acceder si tienes privilegios de súper administrador, se permite aceptar las solicitudes pendientes de creación de una cuenta así como editar y borrar otros perfiles.
    \imagen{imagenes/EditarUsuario}{Editar datos de un usuario}
    También se creó esta pantalla para permitir al usuario editar sus datos.
\end{description}
\section{Ampliación de funcionalidades}
Después de subsanar los errores que existían en la aplicación, se decidió implementar más funcionalidades.\\
\subsection{Introducción de datos desde una base de datos externa}
Para realizar el informe de la coyuntura económica de Burgos se usan múltiples fuentes así que se pensó en una forma de sacar la información de una manera más sencilla para el usuario.\\
Las bases de datos abiertos más utilizadas son: el Instituto Nacional de Estadística, Eurostat y los datos abiertos de la Junta de Castilla y León. En esta sección se hablará de cada una de ellas y se explicará su manera de extraer datos concluyendo con la justificación de la plataforma elegida.
\subsubsection{Instituto nacional de estadística}
Una fuente de la que suelen sacar los datos para el análisis de la coyuntura económica es el Instituto Nacional de Estadística (INE). Por ello, para facilitar la extracción de los datos, se pensó en la posibilidad de permitir seleccionar la información de esta página y que la aplicación la insertara automáticamente en la base de datos.\\
El catálogo de datos de esta plataforma proviene de dos fuentes: la base de datos de difusión Tempus3 y el repositorio de ficheros PC-Axis.\\
La información se distribuye en tablas en las que se permite filtrar los datos según sus características.
 \imagen{imagenes/Ine}{Ejemplo de una tabla del INE}
Una vez se seleccionen los valores que se quieran consultar, los datos aparecen en forma de tablas o distribuidos en un gráfico.
\imagen{imagenes/tablaine2}{Datos de una tabla del INE}
Además de esta manera de visualizar los datos, el Instituto Nacional de Estadística proporciona un servicio que permite descargar la información en formato JSON. Para ello debemos crear una petición en forma de url.\\
\imagen{imagenes/datosine}{Salida de los datos en formato JSON}
 La url para la petición de los datos tiene el formato que aparece en la imagen. 
 \imagen{imagenes/urlJSON}{Estructura de la url}
Los campos que aparecen entre llaves, \{ \}, son obligatorios.\cite{ine:urljson}\\
Los campos que aparecen entre corchetes, [ ], son opcionales y cambian en relación a la función considerada.\\
Descripción de cada uno de ellos
\begin{description}
	\item [idioma] ES para español e EN para inglés. Por defecto está puesto el español.\\
	\item [función] Funciones implementadas en el sistema para poder realizar diferentes tipos de consulta en función del tipo de fuente, Tempus3 o PC-Axis, y del elemento que se quiere obtener.
    \item [inputs] Identificadores de los elementos de entrada de las funciones. Estos inputs varían en base al repositorio del que se extraen los datos.
    \item [parámetros] Los parámetros en la URL se establecen a partir del símbolo ?.\\
    Cuando haya más de un parámetro, el símbolo \& se utiliza como separador.\\
    No todas las funciones admiten todos los parámetros posibles. Por ello haremos una clasificación para explicarlos mejor:  
     \begin{enumerate}
        \item Parámetros comunes a todas las funciones
            \begin{description}
            \item [page] Si hay más de 500 elementos, la consulta se divide en páginas. Esta opción nos permite seleccionar la página que queremos visualizar.
            \item [download] Para descargar el fichero JSON.
            \item [det] Este parámetro da más detalles de la información mostrada.
            \item [tip] Cambia la forma de mostrar la información.
            \end{description}
        \item Parámetros para la petición de datos
            \begin{description}
                \item [date] Filtra los datos por fecha; fecha concreta, lista o rango de fechas.
                \item [nult] Devuelve los últimos n datos. Ejemplo: nult=4 devuelve los 4 últimos datos.
            \end{description}
        \item Parámetros para la obtención de datos y metadatos en base al ámbito geográfico
            \begin{description}
                \item [geo] Con geo = 1 para provincias, municipios u otras desagregaciones y geo = 0 para datos nacionales.
            \end{description}
        \end{enumerate}
\end{description}
\subsubsection{Eurostat}
Eurostat es una base de datos que recoge información a nivel europeo. Tiene un servicio de datos en JSON y en Unicode mediante el cual permite descargar los archivos disponibles en estos dos formatos a través de una petición REST.
\imagen{imagenes/eurostat}{Estructura de una petición REST de Eurostat}\cite{eurostat}
Como se puede ver la estructura es muy parecida a la petición de datos del INE, con una parte fija, otra parte que indica el formato, una que indica el lenguaje y finalmente el código que identifica los datos. También permite filtrar la información.\\
El inconveniente de este servicio es la forma de conseguir el identificador de los datos (\textit{datasetCode} en la imagen).
Los códigos de los conjuntos de datos se pueden encontrar ordenados por temas en la página de Eurostat
\imagen{imagenes/EurostatData}{Conjuntos de datos del Eurostat}
\subsubsection{Datos abiertos de la Junta de Castilla y León}
El portal de Datos Abiertos se enmarca en el proyecto de Gobierno Abierto de la Junta de Castilla y León por el que se pone en marcha el Modelo de Gobierno Abierto de la Junta de Castilla y León junto con la información de transparencia, el espacio de participación ciudadana   y la presencia en redes sociales entre otras actuaciones.\\
Este portal tiene como objetivos aumentar la transparencia, proporcionando mayor información sobre la actividad de la Junta de Castilla y León y conseguir la participación y colaboración de los ciudadanos y empresas, a través de la interlocución con los mismos, de manera que el intercambio de conocimiento y experiencias permita el avance conjunto de la iniciativa pública y privada.\cite{datosab}\\
\imagen{imagenes/apijcyl}{Búsqueda de datos desde la API}
Esta plataforma también permite descargar la información en formato JSON, CSV o Excel. Su funcionamiento es muy simple pero no permite automatizar la extracción de los datos como las bases de datos anteriores.
\imagen{imagenes/datosjcyl}{Algunos conjuntos de datos de la aplicación}
\subsubsection{Conclusión}
Para elegir qué plataforma usar para la extracción de datos se hizo una evaluación de sus características.
\tablaSmall{Comparativa de las plataformas de datos abiertos}{l c c c c}{comparativaplataformas}
{ \multicolumn{1}{l}{} & INE & Eurostat & Datos abiertos JCYL \\}{ 
Posee un sistema de petición de datos por url & X & X & \\
La petición de los datos puede automatizarse & X & & \\
Posee un gran catálogo de datos & X & X & X\\
Facilidad de uso & X & & X\\
Facilidad de extracción de datos & X & &\\
}
Habiendo sopesado las tres opciones, se eligió la plataforma del Instituto Nacional de Estadística por ser la más completa y sencilla de automatizar.
\subsection{Predicción de datos}
Otra funcionalidad interesante que se decidió añadir fue la posibilidad de predecir los datos de una variable.//
Para ello, se utilizaron una serie de algoritmos de predicción de datos. 

\section{Problemas}
A lo largo del desarrollo del proyecto se encontraron problemas que condicionaron el trabajo. En esta sección se expondrán y se explicarán cómo se solucionaron.
\subsection{Problemas de compatibilidad}
Este tipo de problemas tuvieron que ver con Composer.\\
Como ya se ha explicado, Composer es un gestor de dependencias del proyecto. El problema de usar este gestor es que al instalar nuevas librerías era muy frecuente que aparecieran errores de compatibilidad.
\imagen{imagenes/problemaComposer}{Captura de un problema típico al intentar instalar un paquete nuevo}
Para la resolución de estos problemas se han tenido que actualizar, borrar e incluso instalar versiones viejas de las dependencias problemáticas.
\subsection{Problemas en el despliegue del proyecto}
Para la producción del trabajo se han usado muchos servidores locales y on-line como Xampp, Wampp, Docker, etc y todos ellos han supuesto un reto en su correcta configuración.\\
\subsubsection{Xampp}
Xampp ha sido la herramienta más usada para probar el proyecto.\\
La mayoría de los problemas los tuve a la hora de configurar el host virtual.
\imagen{imagenes/vhostxamp}{Configuración de los hosts virtuales}
Al principio, la ruta no era resuelta correctamente por el navegador. Para la solución a este problema hizo falta cambiar el archivo de configuración del servidor de Xampp, y habilitar los hosts virtuales.\\
Además hubo que borrar la caché del navegador ya que cargaba la anterior configuración.
\subsubsection{Docker}
Para usar esta herramienta se usó el contenedor \textit{laradock} que tiene multitud de imágenes para proyectos de Laravel. El principal problema fue el servidor que utilizaba ya que en un principio se usó \textit{nginx} y no funcionaba correctamente. Después se cambió a \textit{apache}.\\
\imagen{imagenes/envlaradock}{Archivo de configuración .env de laradock}
El direccionamiento del proyecto se realiza desde el archivo \textit{.env}.\\
Laradock usa un direccionamiento virtual de los archivos así que esta parte también fue problemática.\\
\subsubsection{Heroku}
Para la presentación de la aplicación se decidió utilizar en Heroku en un principio, que es un hosting gratuito donde se puede alojar tu aplicación web. Hubo varios problemas al instalarlo ya que el proyecto tiene dos partes, la parte del código y la base de datos. Heroku ofrece una base de datos propia pero es de pago, además está en PostgreSQL y la del proyecto está en MySQL.\\
Para solucionar esto se consiguió conectar la aplicación a una base de datos externa totalmente gratuita llamada \href{https://www.db4free.net/}{db4free.net}.\\
Cuando se consiguió que todo funcionara correctamente, se notó la lentitud de la aplicación, pero al ser un hosting gratuito
es normal ya que no nos ofrecen mucha memoria.\\
La aplicación funciona, pero al tener tantos servicios con peticiones agotamos la memoria RAM que nos ofrece Heroku y el rendimiento de la aplicación no es el esperado.\\
Para poder tener un buen rendimiento, sería necesario usar la versión de pago de Heroku, mucho más potente.







\capitulo{6}{Trabajos relacionados}
\section{Api de datos abiertos de la junta de Castilla y León}
El portal de Datos Abiertos se enmarca en el proyecto de Gobierno Abierto de la Junta de Castilla y León por el que se pone en marcha el Modelo de Gobierno Abierto de la Junta de Castilla y León junto con la información de transparencia, el espacio de participación ciudadana   y la presencia en redes sociales entre otras actuaciones.\\
Este portal tiene como objetivos aumentar la transparencia, proporcionando mayor información sobre la actividad de la Junta de Castilla y León y conseguir la participación y colaboración de los ciudadanos y empresas, a través de la interlocución con los mismos, de manera que el intercambio de conocimiento y experiencias permita el avance conjunto de la iniciativa pública y privada.\cite{datosab}\\
\imagen{imagenes/datosjcyl}{Algunos conjuntos de datos de la aplicación}
\section{Instituto nacional de estadística}
La página web del INE cuenta con un funcionamiento muy parecido al que se ha desarrollado en esta aplicación.\\
En ella también se muestra la información de las variables en tablas y en gráficos.\\
\imagen{imagenes/tablaine}{Ejemplo de una tabla del INE}
\imagen{imagenes/graficoine}{Ejemplo de un gráfico del INE}
La página web no va destinada directamente a la creación de tablas y gráficos, sino a dar a conocer que es el INE, sus métodos y proyectos y sus productos o servicios.\\
El INE utiliza varios programas informáticos:
\begin{description}
    \item [Sorolla] Es un sistema de apoyo a la gestión económica de los centros gestores públicos. Entre otras herramientas, se destaca un módulo de elaboración de informes de ejecución presupuestaria (AVANCE). Cuenta con un módulo para elaboración de documentación tributaria relativa a los pagos realizados a personas físicas y jurídicas. Por lo que, ayuda en la elaboración de modelos estadísticos predictivos, sobre la base de la elaboración de los presupuestos.
    \item [Greco] Es un software para el análisis y tratamiento estadístico relativo al turismo, referido a alojamientos, hoteles, casas rurales, etc.
    \item [Celec] Software que se encarga de analizar datos estadísticos relativos al censo de población, extenso y completo. Es el mas utilizado por el INE.
    \item [Ocr] Programa que se encarga de analizar los nacimientos, defunciones, matrimonios y partos de la población. Realiza un estudio de las variaciones de la población española.
    \item [Ida padrón] Software para el análisis y tratamiento de los datos de empadronamiento, se analizan mediante un censo de población junto con un censo de viviendas con una periodicidad de diez años.
    \item [Epf] Aplicación que gestiona las encuestas que hacen referencia a los presupuestos familiares.
\end{description}

\capitulo{7}{Conclusiones y Líneas de trabajo futuras}
\section{Conclusiones técnicas}
Considero que casi todos los objetivos impuestos se han cumplido satisfactoriamente. Los errores de la aplicación se han solucionado y además se han conseguido añadir las nuevas funcionalidades propuestas:
\begin{itemize}
    \item Extracción de datos desde el INE.
    \item Predicción de datos y representación gráfica.
    \item Edición de usuarios.
    \item Extracción de las tablas a formato Excel.
\end{itemize}
Además se ha cumplido una de las principales exigencias por la parte del cliente que fue la subida de la aplicación al servidor de la UBU.
El objetivo que cumplí desde el principio del trabajo fue el seguimiento con la herramienta Github. Tuve muchos problemas para configurar esta herramienta con mi repositorio y con las otras aplicaciones de integración continua que he usado.
\section{Conclusiones personales}
Puedo concluir diciendo que este ha sido un proyecto que me ha supuesto mucho esfuerzo porque no había trabajado nunca con este entorno de desarrollo, ni este lenguaje ni sabía nada sobre el desarrollo y despliegue de aplicaciones web.\\
Aunque hubo un tiempo en el que me arrepentí de haber elegido un tema del que no tenía conocimientos previos, tengo que reconocer que he aprendido mucho y esta experiencia me ha enseñado a ser autodidacta y a cómo afrontar proyectos futuros en solitario sin nadie que me guíe.
\section{Líneas de trabajo futuras}
Tengo algunas ideas sobre posibles cambios o mejoras respecto al proyecto.
\begin{itemize}
    \item Mejorar la extracción de datos desde el INE ya que hay algunos conjuntos de datos que no es capaz de reconocer.
    \item Migrar la aplicación a una app móvil.
    \item Importar datos desde otras plataformas mencionadas anteriormente como Eurostat o los datos abiertos de la Junta de Castilla y León.
    \item Sería interesante poder exportar todos los datos de las tablas y de los gráficos a otro tipo de ficheros como por ejemplo “.pdf”,“.doc”, etc.
\end{itemize}


\bibliographystyle{plainnat}
\bibliography{bibliografia}

\end{document}
